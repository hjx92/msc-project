\documentclass{article}
\usepackage{graphicx} % Required for inserting images
\usepackage{biblatex}
\usepackage{csquotes}
\addbibresource{references.bib}

\title{A three-dimensional rail shooter in PICO-8: project proposal}
\author{Harry Jackson}
\date{June 2025}

\begin{document}
\maketitle

\begin{abstract}
The rail shooter was one of the earliest attempts to bring video
games into the third-dimension - first through the use of pseudo-3D
scaling effects, and later through the use of real-time polygonal graphics.

This project seeks to explore this category of interactive software through a
modern implementation for PICO-8 - a ``fantasy console" which replicates the
computational and data constraints of historical system hardware, on which
developers typically eschewed polygonal rendering for simpler 2D effects.
\end{abstract}

\section{Introduction}

// insert picture-setting and general project introduction

\subsection*{Objectives and challenges}

\subsubsection*{Lightweight game engine design}
PICO-8 is a fantasy console, not a game engine. It offers basic utilities to execute a game loop -
namely a per-frame update function, and functions to draw sprites, draw primitives, read inputs, and
play sounds. As such, one major aspect of this project will be to implement a bespoke engine that is
appropriate to the contraints of PICO-8, but which facilitates a rational approach to development of
the game proper and avoid common game development pitfalls.

One countervailing constraint is to avoid the temptation to implement too much - the purpose of the
project is to develop a single game, not to create a general purpose game engine for PICO-8 that is
extensible across a variety of game designs in diverse genres.\cite{gregory}

\subsubsection*{Real-time 3D graphics in a resource-constrained system}
The imposed limitations of the PICO-8 fantasy console - which can execute approximately four million
virtual instructions per second, and harshly limits program length and available data in ``carts" -
are inherently challenging for 3D rendering.

The coarse 128x128 pixel grid of the PICO-8 system, and its small 16-tone colour palette, also create
challenges for the effective implementation of 3D graphics - limiting the system's native ability to
render small polygons and to effectively model even simple lighting through flat-shading.

As a result, strategies will need to be devised for (1) efficiently rendering simple polygonal
models, (2) creating expressive polygonal models that do without complex geometry, (3) intelligently
reducing stored geometric data by, for example, exploiting symmetry, and (4) effectively shading
polygons through methods other than changes in RGB colour value, such as dithering.

\cite{akenine-möller-etal}

\subsubsection*{Collision detection}
\cite{ericson}

\subsubsection*{Irregular collisions}
The inclusion of a real-time three-dimensional boss with interactable geometry called for a second
and more specialised collision detection system, checking for intersections between triangular faces
in 3D space and rays defining the motion of projectiles, rather than simple checks for overlapping
bounding volumes.

Such an approach was required to execute the design of a boss enmeshed in a three-dimensional protective
shell - a regular icosahedron - which the player must destroy by strafing in a circular motion to
identify and successfully shoot weak points under time constraint.

To achieve this, I made use of the Möller–Trumbore ray-triangle intersection algorithm.\cite{möller–trumbore}

\subsubsection*{Marrying 2D and 3D graphics in a low-resolution environment}
The game will attempt to intelligently combine 2D and 3D rendering techniques to limit
the computational load of 3D elements, and facilitate more interesting designs that would
be virtually impossible with a full 3D approach due to aforementioned system constraints.

One of the key challenges will therefore be combining 2D and 3D rendering in a cohesive game - 
defining a coherent relationship between 3D world space and 2D screen space for
gameplay interactions, and effectively rendering 2D sprites at depth.

\subsubsection*{Object-oriented game programming in Lua}
The enforced programming language for the PICO-8 system is Lua, a multi-paradigm programming
language created in 1993 by Roberto Ierusalimschy, Waldemar Celes, and Luiz Henrique de Figueiredo.

The language uses objects, but it does not have the concept of a class - instead facilitating
object-oriented programming through object prototyping and the use of shared inheritance
from metatables.\cite{ierusalimschy} This design philosphy is explained by Ierusalimschy in his
primer \textit{Programming in Lua}:

\begin{displayquote}
A class works as a mold for the creation of objects. Several OO languages offer the concept of class. In such languages, each object is an instance of a specific class. Lua does not have the concept of class; each object defines its own behavior and has a shape of its own.
\end{displayquote}

This language feature will demand careful adaptation of established game programming design
patterns, which have grown out of industrial practice in the more traditionally class-based
object-oriented \textit{lingua franca} of game development, C++.\cite{nystrom}

\subsubsection*{Avoiding known genre pitfalls}
% Common criticisms of the rail shooter genre
Seminal works in the genre were largely used as technical showpieces, highlighting incipient
hardware technologies like Sega's Super Scaler arcade boards, Nintendo's Super FX graphics chip, and
new-to-market 3D consoles like the Sega Saturn and Nintendo 64.

The genre fell out of favour, and saw much reduced commercial production, as 3D games became more
common and more expansive approaches to interaction in 3D space were opened by the dawn of
dual-analogue input.

Thus, one of the core challenges of this project will be a foundational game design challenge:
creating a game in this genre that avoids common criticisms levelled at the genre, for its
restricted and overly scripted gameplay.

\subsection*{Project scope}
The project aims to deliver a feature-complete and content-complete game,
sized appropriately for the chosen medium - consisting of three complete
levels, each of which introduce unique enemy patterns, a unique boss, and
mechanisms for successfully incrementing game challenge in a satisfying way.

Because of the constraints of the PICO-8 fantasy console and the computational
demands of 3D rendering, the game aims to run in the system's default 30Hz mode,
rather than its optional 60Hz refresh mode.

\subsection*{Motivation}
PICO-8 is a popular platform for retro-style game development, game prototyping, and game jamming.
3D rendering on the platform has been a persistent interest of PICO-8 enthusiasts, who have built
rendering engines and technical demos on this theme - including rudimentary flight simulators,
wire-frame platformers, and de-makes of popular games like \textit{Doom} and \textit{Star Fox}.

This project seeks to advance beyond these exercises with a full game development project, implementing
an original interactive game that draws on known mechanics of the genre (both from the period PICO-8 seeks
to emulate, and beyond) and builds on top of them.

Despite the known criticisms of the genre outlined above, many games of this
class have retained an audience among retro game enthusiasts. Significant titles have been
remade and remastered for modern systems - most notably \textit{Rez Infinite}, which has
recently found a new home on virtual reality platforms like the Oculus Quest and PSVR2,
where it has been well-received.

% Significance of Pico-8 as prototyping tool
\subsection*{Related work}
Relevant related commercial software can broadly be defined
chronologically and broken down into three categories:

\begin{enumerate}
    \item 8- and 16-bit rail shooters such as 
    \textit{Space Harrier} (arcade, various consoles),
    \textit{The 3-D Battles of WorldRunner} (Nintendo Entertainment System), and 
    \textit{Star Fox} (Super Nintendo),
    which demonstrate software built within similar constraints;
    
    \item Early 3D rail shooters with limited or no analogue input such as 
    \textit{Panzer Dragoon} (Sega Saturn) and
    \textit{Star Fox 64} (Nintendo 64),
    which implement diverse approaches to navigating three-dimensional space with a directional-pad or
    single analogue stick;
    
    \item Mature 3D rail shooters with modern gamepad inputs such as 
    \textit{Rez} (Dreamcast, modern systems) and
    \textit{Panzer Dragoon: Orta} (Xbox),
    which point to non-technical design innovations that could be explored in this project.
\end{enumerate}

% Brief outline of genre and development

\section{Implementation}
\subsection*{Implementation plan}

\subsection*{Choice of technology}
% Refer to motivations already outlined
\subsection*{Comparison to related work}

% Detailed comparison of Pico-8 and industrial implementation on
% fixed console hardware, with less constrained resources

\subsection*{3D capability testing}
% Not the correct place for this material - simply placed here for safe keeping...
\begin{center}
\begin{tabular}{r|c c c c c c c c c c}
     Cubes drawn & 0 & 1 & 2 & 3 & 4 & 5 & 6 & 7 & 8 & 9 \\
     \hline
     CPU utilisation & .37 & .39 & .41 & .43 & .46 & .47 & .49 & .50 & .52 & .53
\end{tabular}
\end{center}

\section{Evaluation}

\section{Time Plan}
\subsection*{Prototyping}
Prototyping began on May 19, following initial conversations with Dr Lapinskas. This process involved
rudimentary paper prototyping and a naive software implementation - developing familiarity with Lua,
sketching out the core gameplay loop, and experimenting with 3D rendering implementations. 

These software prototypes consisted of (1) a fully 2D implementation of the core game,
drawn with primitives and using naive approximations to simulate perspective,
(2) a standalone 3D renderer drawing a cube, with rotations controlled by gamepad input,
and (3) a 3D implementation of the core game, with previously faked elements
now given proper positions in 3D space.

\subsection*{Full implementation}
My work plan is based around three key software development milestones:
\begin{itemize}
    \item \textbf{A functionally-complete game} by the end of June,
    with appropriate software architecture\cite{nystrom}
    and perspective-correct real-time 3D rendering\cite{gambetta};
    \item \textbf{A content-complete game} by the end of July,
    with three levels populated by progressively more intricate enemy, obstacle, and boss designs;
    \item \textbf{A complete game} that is artworked and soundtracked
    by project submission at the start of September, with much of August devoted to
    \textbf{report writing}.
\end{itemize}
I do not have any planned vacation during the summer, so will be working on the
project during weekdays for the duration of the unit.

\printbibliography

\end{document}
