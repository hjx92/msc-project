\documentclass{article}
\usepackage{graphicx} % Required for inserting images
\usepackage{biblatex}
\usepackage{csquotes}
\addbibresource{references.bib}
\renewcommand{\abstractname}{Executive Summary}

\title{A three-dimensional rail shooter in PICO-8}
\author{Harry Jackson}
\date{September 2025}

\begin{document}
\maketitle

\begin{abstract}
The rail shooter was one of the earliest attempts to bring video
games into the third-dimension - first through the use of pseudo-3D
scaling effects, and later through the use of real-time polygonal graphics.

This project seeks to explore this category of interactive software through a
modern implementation incorportating polygonal rendering for PICO-8 - a
``fantasy console'' which replicates the computational and data constraints of
historical system hardware on which developers typically eschewed polygonal
rendering for simpler 2D effects.
\end{abstract}

\section*{Acknowledgements}

\tableofcontents

\section{Introduction, aims, and objectives}
[To be outlined and written after composition of rest of main body]


\section{Background and context}

\subsection{Genre background}
\subsubsection*{Historical outline}
Give broad overview of significance of rail shooters in the development of approaches
to 3D game design and chart their development with a focus on:
\begin{itemize}
   \item\textit{Space Harrier} and the 2D Super Scaler approach;
   \item\textit{Star Fox} as an exemplar for 16-bit polygonal 3D with
   cartridge storage limitations;
   \item\textit{Panzer Dragoon} as a 32-bit polygonal 3D game on optical media;
   \item and \textit{Rez} as a mature rail shooter with continued life in VR and
   ongoing aesthetic significance.
\end{itemize}
\subsubsection*{Gameplay taxonomy}
Describe the basic structure of a 3D rail shooter with reference to the above examples,
outlining their various approaches to core gameplay elements like enemy design, player
movement, shooting and targetting, significant action set-pieces, and their use of
fixed perspective or a malleable 3D camera.
\subsubsection*{Genre pitfalls}
Evidence common criticisms of 3D rail shooters as ``glorified technical demos'' with
limited room for gameplay innovation and discuss their waning significance in
commercial game design - with appropriate counter-examples of cult following for
balance.

\subsection{PICO-8}
\subsubsection*{Introduction to PICO-8}
Discuss ethos of PICO-8 as a ``fantasy console'' and outline its technical
specification - emphasis on compute power, memory limitations, screen resolution,
player inputs, and token limits - and indicate some of the implications of these
constraints on project scope.
\subsubsection*{PICO-8 as prototyping tool}
Discuss benefits of using PICO-8 as a tool for rapid creation and iteration of
game prototypes, compare and contrast to full 3D engines like Unity, Godot and Unreal,
describe suitability given fixed time constraints of an individual summer
project, and point to games successfully prototyped in PICO-8 (most notably
\textit{Celeste}).
\subsubsection*{Sprite-scaling functionality}
Discuss PICO-8's given functionality for sprite drawing and scaling and explain how
these would facilitate development of a low-fidelity game utilising a pseudo-3D
Super Scaler approach.

Discuss \textit{Star Fox} as exemplar polygonal rail shooter built within similar
constraints and make critical comparison between PICO-8 platform specification and
the Super Nintendo with on-cartridge Super FX graphics chip.

Discuss extensibility of
PICO-8's functions for drawing primitives (eg rectfill and line) for triangle
rasterisation and polygonal rendering.
\subsubsection*{3D experimentation in PICO-8}
Broad overview of existing experimentation with polygonal 3D in PICO-8 with a
focus on \textit{PICO-Fox} as closest analogue. 

Discuss exposed limitations of approach - particularly heavy cost of geometry data
for 3D models, and limitations imposed on design by use of polygonal enemies as
standard (ie minimum size constraint forced by low readability of small 3D models
on a coarse pixel grid).
\subsubsection*{Mitigation strategies for constraints}
Discuss apparent mitigation stategies to work effectively with PICO-8 in light of
the known constraints of the system. Emphasis on:
\begin{itemize}
   \item aggressive use of OOP to eliminate code dupication and preserve token count;
   \item use of standard 30Hz display mode rather than higher-refresh 60Hz mode, to
   trivially double frame budget when drawing and manipulating 3D models;
   \item a blended 2D/3D approach that reserves polygonal rendering for the player
   sprite and for boss encounters, and allows the flexibility and low cost of
   sprite-based gameplay throughout the rest of the core gameplay loop;
   \item possible growth into a ``multicart'' game, in violation of a basic intended
   constraint of the system, as a last resort to deal with token limits.
\end{itemize}


\section{Implementation}

\subsection{Paper prototyping, software prototyping, and 3D rendering test}
\subsubsection*{Basic player actions using primatives}
Referring back to the gameplay taxonomy conducted in the background section, explain
how the most significant user actions were distilled down to fit a six-input
gamepad. Showcase how this was modelled using a flat paper prototype and outline
how this was translated into a software prototype using primitive shapes to model
the player and randomly spawned enemies, with movement and interactions being
directly translated into screen-space.

Highlight issues already identified at this early stage of prototyping - such as
player model blocking, the relationship between the player reticle and the destination
of projectiles in hypothecated 3D space, the need for limitations on enemy
target locking, and the need for game objects to have a meaningful existence in 3D
space.
\subsubsection*{Simple 3D renderer with model scaling and rotations}
Describe the contruction of a simple 3D renderer to prove suitability of PICO-8 for
the proposed project. Expose and discuss my implementation of triangle rasterisation
in PICO-8 and briefly explain the benefits of decomposing 3D shapes into triangles
rather than quads. Discuss related issues in polygonal rendering such as triangle
sorting and backface culling.

Showcase the results of the ``sanity test'', in which multiple
cubes are drawn to the screen progressively in order to understand the relative
computational cost of tracking and drawing 3D geometry in PICO-8 and the feasibility
of real-time polygonal rendering given system constraints and my existing quality
of implementation.

\subsection{3D conversion of software prototype}
\subsubsection*{Perspective-correct 3D projection and scaling}
Discuss the application of 3D point projection to the corners of 2D sprite objects
to achieve perspective correct drawing and scaling of 2D elements using PICO-8's sspr
function. Highlight use of small camera movements (spatial displacement in x and y)
aligned to player sprite movement to heighten illusion of depth along the z-axis.
\subsubsection*{Real-time 3D collisions}
Discuss implementation of cuboid bounding volumes to test for collisions, using
height, width, and depth data paired if three-dimensional positions to check for
intersection of volumes. Discuss the need for axis-alignment to significantly
simplify calculations, and the design implications of this with regard to rotational
symmetry of objects depicted (ie disc-shaped saucers, and spherical projectiles)
\subsubsection*{Graphical effects with 2D sprites and particles}
Describe the implementation of 2D sprites to replace primitive shapes used in earlier
iterations of the prototype. Discuss the design considerations that led to the
adoption of easily identifiable flying saucer and drone designs. Discuss simple
graphical effects used to eg light up flying saucer lights, or spin rotors of
enemy drones. Explain implementation of enemy destruction effects - namely the
implementation of a simple particle system over the bed of an explosion animation
efficiently faked by a combination of progressive scaling and random flipping of a
non-symmetrical sprite.

\subsection{Full implementation}
\subsubsection*{Settled software architecture and object heirarchy}
Discuss the software architecture and walk through the significant parts of the
class diagram that was drafted at the end of software prototyping. Explain how the
design was informed by prototyping and broader game systems architecture reading.
Describe how game flow is controlled within the architecture. Descibe how game object
subclassing of sprite objects and polygonal objects facilitates the blended
2D/3D rendering approach outlined earlier in a polymorphic way. 
\subsubsection*{The game loop}
Describe the core gameplay loop and give a comprehensive tour of the update and draw
functions in main.lua that execute it, explaining the rationale for event and draw
sequencing. Discuss the z-sort implementation in draw, comparing to alternative sorting
algortithms, and describe how polymorphic class design enables heterogeneous elements
to be drawn together in the correct depth ordering.
\subsubsection*{Camera rotations in 3D space}
Discuss the implementation of camera rotations around a point in 3D space for the
creation of a ``boss mode'' which involves strafing around a polygonal enemy. Highlight
the implications of this change on the position and orientation of the player model,
which can no longer exist naively in a 2D plane facing along the z-axis but which
must now be able to exist in a rotational relationship to a focal point in the map,
shared with the camera. Discuss the relationship between in-frame player movement,
the implied field of view of a 1x1 canvas at single unit distance from pinhole, and
camera rotation around a point. Discuss the per-object rotations required to execute
such camera movements, density of spawned scenery, and concerns around computational
load.
\subsubsection*{Ray-triangle collisions with the Möller–Trumbore algorithm}
Discuss the need for more complex ray-triangle intersection computations to effectively
test for collisions around the boss enemy, since the intended design requires per-face
collision detection on a regular icosahedron whose faces cannot be satisfactorily
split into non-overlapping cuboid bounding volumes that adequately approximate
collisions.

\section{Results}
\subsection{Think-aloud testing feedback}
Discuss feedback gathered by think-aloud testing sessions, and highlight adaptations
to the software intended to answer constructive criticisms. Possible subjects of
discussion:
\begin{itemize}
   \item implementation of on-player HUD to answer lack of player attention on vital
   gameplay statistics, ie remaining life and state of lock-on targetting;
   \item redesign of pick-up sprites to better differentiate power-ups from
   damage-inducing projectiles;
   \item introduction of narrative splash screen to justify on-screen action, which
   was seen as random and unexplained.
\end{itemize}
Also discuss positive feedback gathered during sessions, such as satisfaction
with the game controls, hit detection, and overall aesthetic of the game, and comfort
with depth perception despite the low-resolution pixel grid.
\subsection{Formal NASA-TLX testing}
TBD


\section{Evaluation}


\section{Further work}


\section{Conclusion}
[To be outlined and written after composition of rest of main body]

\printbibliography

\end{document}